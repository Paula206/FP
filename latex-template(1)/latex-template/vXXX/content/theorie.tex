\section{Theorie}
\label{sec:Theorie}

%\subsubsection{Szintillator}
%Die Messung wird mit Hilfe von einem Szintillatosdetektor durchgeführt. Dieser macht sich den Effekt der Lumineszenz zu nutzte. 
%Lumineszenz ist die Emission von Licht mit einem charakteristischen Spektrum. Diese wird durch die Absorption von ionisierter Strahlung erzeugt. 
%Dabei wird zwischen zwei Arten von Szintillatoren unterschieden. Es gibt den organischen und anorganischen Szintillator.
%In diesem Versuch wird ein organischer Szintillator benutzt.  
%Bei diesen wird durch Hilfe der hinzugefügten Energie der einfallenden Strahlung die Atome in einen angeregten Zustand gehoben. 
%Wenn das Atom wieder in seinen normalen Zustand zurück geht, wird Energie in Form von Photonen abgegeben.
%
%\subsubsection{Photonenvervielfacher}
%Die Abgegebenen Photonen werden mit Hilfe eines Photonenvervielfacher (PMT) in einen elektrischen Impuls umgewandelt. 
%Dieser besteht aus einer Photokathode, einen Verstärkersystem, das aus mehreren Dynoden besteht und einer Anode. 
%Das Licht, welches durch den Szintillator entstanden ist, trifft durch das PMT-Fenster auf die Photokathode. 
%Dort werden durch den Photoeffekt Elektronen emittiert. Diese werden durch ein Elektrisches Feld auf die erste Dynode fokussiert. 
%Beim Auftreffen auf die Dynoden wird ein Elektron durch die Emission von Sekundärelektronen vervielfacht. Diese werden darauf auf eine nächste Dynode beschleunigt. 
%Nach mehreren Dynoden, wo diese wieder vervielfacht werden, treffen die Elektronen auf eine Anode.
%
%\noindent So wird ein Lichtblitz in einen elektrischen Impuls umgewandelt. 
%\subsubsection{Aufbau}
%Der Aufbau des Versuchs ist in Abbildung 1 dargestellt.  An den organischen Szintillator im zylinderförmigen Edelstahltank befinden sich zwei PMT angeschlossen. 
%Von diesen geht der Elektrische Impuls. Dieser durchläuft ein Schaltsystem bis es in einen Zeit-Amplituden-Konverter (TAC) gemessen wird. 
%Der TAC gibt einen Spannungsimpuls ab, dessen Amplitude proportional zu der Zeit zwischen zwei einlaufenden Impulsen ist. 
%Dieses Signal wird dann an einen Vielkanalanalysator weitergeleitet. Der Vielkanalanalysator sortiert die einkommenden Impulse und histogrammiert diese.
%Die Daten werden dann an einen Computer geleitet und gespeichert.
%
%Da es unterschiedliche Störprozesse gibt werden unterschiedliche Bauelemente eingebaut, um diese zu unterdrücken.
%Zum einen wird auch durch andere einfallende Teilchen ein Impuls ausgelöst oder auch durch eine spontane Elektronenemission der Photokathode. 
%Diese Signale werden durch einen Diskriminator herausgefiltert. Nach dem Diskriminator werden die Signale von den beiden PMT im Koinzidenz zusammengeführt. 
%Dieser gibt nur ein Signal weiter, wenn von beiden PMT gleichzeitig ankommen.
%
%\subsubsection{Logische Schaltung}
%Da viele Myonen nicht im Szintillator zerfallen, werden viele Signale ausgelöst, die nicht gezählt werden sollen. Die Schaltung misst die Zeit zwischen zwei Signalen. 
%Dabei starten der erste Impuls die Zeitmessung des TACs und das zweite endet diese. 
%Nun darf nur ein Impuls die Zeit stoppen, welcher beim Myonenzerfall entstanden ist und nicht eins, das durch ein anderes Myon entstanden ist. 
%Daher muss nach einer gewissen Zeit, der Suchzeit, die Messung abgebrochen werden und mit einem neuen Impuls anfangen.
%Dazu wird eine monostabile Kippstufe und zwei AND-Gatter verbaut. 
%Das Startsignal stammt aus dem einem AND-Gatter, welches das Signal aus der negierten Kippstufe und die Koinzidenz vergleicht.  
%Von der Koinzidenz führt dasselbe Signal über eine 30 ns Verzögerungsleitung zur Kippstufe und zu einem zweiten AND-Gatter. 
%Das zweite Signal fürs AND-Gatter stammt von der Kippstufe.  Das Signal aus diesem AND-Gatter ist für den Stopimpuls zuständig.
%
%Wenn der erste Impuls aus einer Koinzidenz kommt, wird der TAC gestartet, da die beiden Signale, die das erste AND-Gatter erreichen True sind.  
%Nach 30ns kippt die Kippstufe und ein weiteres Signal stoppt die Messung. Der Zeitraum in dem die Kippstufe gekippt bleibt entspricht der Suchzeit. 
%Wenn diese Zeit verstrichen ist stoppt ein weitere Impuls den TAC nicht sondern startet die Messung neu.
%
%\section{Durchführung}
%Der Aufbau wird wie bei Abbildung 1 aufgebaut und justiert. Dabei werden zwischen den Diskriminator und den Koinzidenz Verzögerungsleitungen 
%geschaltet. Diese werden so eingestellt das die beiden Impulse, die von einem Myon ausgelöst werden, 
%gleichzeitig bei der Koinzidenz erreichen. Außerdem wird die Diskriminatorschwelle so eingestellt, dass bei beiden Messkanälen 
%die Impulsrate ungefähr gleich ist. 


%\subsection{Zielsetzung}
%Ziel des Versuches ist die Lebensdauer von Myonen zumessen.
%
%\subsection{Myonen}
%Myonen sind Teilchen, die Teil der sekundären kosmischen Strahlung sind. Die Myonen, die die Erdoberfläche erreichen, 
%entstehen durch den Zerfall von geladenen Pionen. 
%
%\begin{align*}
%    \pi ^+ \to \mu^+ + \nu_{\mu}
%    \pi ^- \to \mu^- + \overline{\nu_{\mu}}
%\end{align*}
%
%Diese entstehen durch Protonen,
%die aus dem Universum stammen und wenn sie in die Erdatmosphäre treten mit den Luftmolekülen wechselwirken.
%Myonen sind wie Elektronen und Tauon, Leptonen. Leptonen unterliegen der schwachen Wechselwirkung und sind Fermionen, 
%das heißt sie besitzen den Spin $\frac{1}{2}$ .
%
%Beim Zerfallsprozess müssen jeweils der Impuls und die Quantenzahlen erhalten sein.
%
%Myonen zerfallen gemäß: 
%
%\begin{align*}
%    \mu ^+ \to \e^+ + \overline{\nu_{\mu}} +\nu_{e}
%    \mu ^- \to \e^- + \overline{\nu_{e}} + \nu_{\mu}
%\end{align*}
%
%
%\noindent Um die Lebensdauer eines Myons zumessen wird ein Szintillator benutzt.  Wenn nun ein Myon in den Szintillator fällt, gibt es ein Teil seiner 
%kinetischen Energie ab. Dadurch wird ein Lichtblitz ausgelöst und in einen elektrischen Impuls umgewandelt. 
%Um nun die Lebensdauer der Myonen zumessen werden die Myonen benötigt, die ihre gesamte kinetische Energie verlieren und im Szintillator verbleiben. 
%Wenn diese eintreten wird, ein Signal ausgelöst. Nach einer gewissen Zeit zerfällt das Myon. Dabei entsteht ein Elektron, das auch mit dem Szintillator wechselwirkt 
%und daher ein Signal auslöst.
%
%
%\subsection{Lebensdauer}
%Die Lebensdauer von Myonen ist ein stochastischer Prozess. 
%Jedes Myon hat eine unterschiedliche Lebensdauer. Daher wird die Wahrscheinlichkeit dW das ein Myon im infinitesimalen Bereich zerfällt benötigt. 
%Es lässt sich der Zusammenhang
%
%\begin{align*}
%dW = \lambda dt
%\end{align*}
%
%\noindent herstellen. Dabei ist $\lambda$ die charakteristische Zerfallskonstante. 
%Also hängt die Zerfallswahrscheinlichkeit nicht vom Alter eins individuellen Teilchens ab. 
%Das Myon unterliegt keinen Alterungsprozess. Daraus ergibt sich der weitern Zusammenhang
%\begin{align*}
%dN = -N dW =-\lambda N dt
%\end{align*}
%
%\noindent mit $dN$ der Zahl der Teilchen, die im Zeitraum dt Zerfallen sind, wenn die Anzahl $N$ Teilchen beobachtet werden.  
%Für die Lebensdauer folgt daraus die Exponentialverteilung auf dem Intervall $t$ bis $dt$.
%
%\begin{align*}
%dN(t) = N_0 \lambda e^- \lambda t dt
%\end{align*}
%
%Der Erwartungswert von dieser Verteilung ist die charakteristische Lebensdauer der Myonen $\tau$.
%
%\begin{align*}
%\tau = \int_{0}^{1} \lambda t e^-\lambda \, \symup{d}t = \frac{1}{\lambda}
%\end{align*}
%
%
%
