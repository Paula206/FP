\section{Durchführung}
\label{sec:Durchführung}
%\subsection{Zielsetzung}
%Ziel des Versuches ist die Lebensdauer von Myonen zumessen.
%
%\subsection{Myonen}
%Myonen sind Teilchen, die Teil der sekundären kosmischen Strahlung sind. Die Myonen, die die Erdoberfläche erreichen, 
%entstehen durch den Zerfall von geladenen Pionen. 
%
%\begin{align*}
%    \pi ^+ \to \mu^+ + \nu_{\mu}
%    \pi ^- \to \mu^- + \overline{\nu_{\mu}}
%\end{align*}
%
%Diese entstehen durch Protonen,
%die aus dem Universum stammen und wenn sie in die Erdatmosphäre treten mit den Luftmolekülen wechselwirken.
%Myonen sind wie Elektronen und Tauon, Leptonen. Leptonen unterliegen der schwachen Wechselwirkung und sind Fermionen, 
%das heißt sie besitzen den Spin $\frac{1}{2}$ .
%
%Beim Zerfallsprozess müssen jeweils der Impuls und die Quantenzahlen erhalten sein.
%
%Myonen zerfallen gemäß: 
%
%\begin{align*}
%    \mu ^+ \to \e^+ + \overline{\nu_{\mu}} +\nu_{e}
%    \mu ^- \to \e^- + \overline{\nu_{e}} + \nu_{\mu}
%\end{align*}
%
%
%\noindent Um die Lebensdauer eines Myons zumessen wird ein Szintillator benutzt.  Wenn nun ein Myon in den Szintillator fällt, gibt es ein Teil seiner 
%kinetischen Energie ab. Dadurch wird ein Lichtblitz ausgelöst und in einen elektrischen Impuls umgewandelt. 
%Um nun die Lebensdauer der Myonen zumessen werden die Myonen benötigt, die ihre gesamte kinetische Energie verlieren und im Szintillator verbleiben. 
%Wenn diese eintreten wird, ein Signal ausgelöst. Nach einer gewissen Zeit zerfällt das Myon. Dabei entsteht ein Elektron, das auch mit dem Szintillator wechselwirkt 
%und daher ein Signal auslöst.
%
%
%\subsection{Lebensdauer}
%Die Lebensdauer von Myonen ist ein stochastischer Prozess. 
%Jedes Myon hat eine unterschiedliche Lebensdauer. Daher wird die Wahrscheinlichkeit dW das ein Myon im infinitesimalen Bereich zerfällt benötigt. 
%Es lässt sich der Zusammenhang
%
%\begin{align*}
%dW = \lambda dt
%\end{align*}
%
%\noindent herstellen. Dabei ist $\lambda$ die charakteristische Zerfallskonstante. 
%Also hängt die Zerfallswahrscheinlichkeit nicht vom Alter eins individuellen Teilchens ab. 
%Das Myon unterliegt keinen Alterungsprozess. Daraus ergibt sich der weitern Zusammenhang
%\begin{align*}
%dN = -N dW =-\lambda N dt
%\end{align*}
%
%\noindent mit $dN$ der Zahl der Teilchen, die im Zeitraum dt Zerfallen sind, wenn die Anzahl $N$ Teilchen beobachtet werden.  
%Für die Lebensdauer folgt daraus die Exponentialverteilung auf dem Intervall $t$ bis $dt$.
%
%\begin{align*}
%dN(t) = N_0 \lambda e^- \lambda t dt
%\end{align*}
%
%Der Erwartungswert von dieser Verteilung ist die charakteristische Lebensdauer der Myonen $\tau$.
%
%\begin{align*}
%\tau = \int_{0}^{1} \lambda t e^-\lambda \, \symup{d}t = \frac{1}{\lambda}
%\end{align*}
%
%
%