\section{Durchführung}
\label{sec:Durchführung}


\subsection{Bestimmung der Zeitkonstanten}
%\begin{figure}
%    \centering
%    \includegraphics{schalt2.png}
%    \caption{Schaltkreis zur Bestimmung der Zeitkonstanten}
%    \label{fig:schalt2}
%\end{figure}

\noindent Für die Bestimmung der Zeitkonstante $RC$ wird der obere Versuchsaufbau \ref{fig:schalt2} benutzt.
Hier wird nun das Aufladen oder Entladen eines Kondensator der Kapazität $C$ über einen Widerstand 
$R$ betrachtet. Dafür wird eine Rechteckspannung angelegt und dann können
 Werte für den Auf- und Entladevorgang abgelesen werden.

\subsection{Bestimmung der Frequenzabhängigkeit der Kondensatoramplitude}
%\begin{figure}
%    \centering
%    \includegraphics{schalt4.png}
%    \caption{Schaltkreis zur Untersuchung der Frequenzabhängigkeit der Kondensatorspannung}
%    \label{fig:schalt4}
%\end{figure}

\noindent Nun wird der Stromkreis \ref{fig:schalt4} verwendet. Dann wird eine Sinusspannung angelegt und die 
Kondensatorspannung in Abhängigkeit von der Frequenz bestimmt.

\subsection{Bestimmung der Phasenverschiebung}
Es wird der selbe Stromkreis wie in dem vorherigen Versuchsteil genutzt.
%\begin{figure}[H]
%    \centering
%    \includegraphics[width=0.5\textwidth]{latex/images/Zweistrahloszillograph.PNG}
%    \caption{Messung der Phasenverschiebung zwischen zwei Spannungen mit dem Zweistrahloszillograph.\protect \cite{V353}.}
%    \label{fig:1}
%\end{figure}

\noindent Nun werden a und b bestimmt, durch die sich dann $\phi = \dfrac{a}{b} 2\pi$ berechenen lässt.