\section{Theorie}
\label{sec:Theorie}

\subsection{Zielsetzung}
Ziel des Versuches ist die Lebensdauer von Myonen zumessen.

\subsection{Myonen}
Myonen sind Teilchen, die Teil der sekundären kosmischen Strahlung sind. Die Myonen, die die Erdoberfläche erreichen, 
entstehen durch den Zerfall von geladenen Pionen. 

\begin{align*}
    \pi ^+ \to \mu^+ + \nu_{\mu} \\
    \pi ^- \to \mu^- + \overline{\nu_{\mu}}
\end{align*}

\noindent Pionen entstehen durch Protonen,
die in der Erdatmosphäre mit den Luftmolekülen wechselwirken. Da Pionen nur eine sehr geringe Lebensdauer besitzen, zerfallen sie 
in etwa 10 km Höhe zu Myonen.
Myonen sind wie Elektronen und Tauon, Leptonen. Leptonen unterliegen der schwachen Wechselwirkung und sind Fermionen, 
das heißt sie besitzen den Spin $\frac{1}{2}$.
Myonen haben auch nur eine endliche Lebensdauer und zerfallen in unteranderen einen Elektron.
Beim Zerfallsprozess müssen jeweils der Impuls und die Quantenzahlen erhalten sein.

\noindent Myonen zerfallen gemäß: 

\begin{align*}
    \mu ^+ \to e^+ + \overline{\nu_{\mu}} +\nu_{e} \\
    \mu ^- \to e^- + \overline{\nu_{e}} + \nu_{\mu}
\end{align*}


\noindent Für die Messung der Lebensdauer eines Myons wird ein Szintillator benutzt.  Wenn nun ein Myon in den Szintillator fällt, gibt es ein Teil seiner 
kinetischen Energie ab. Dadurch wird ein Lichtblitz ausgelöst und in einen elektrischen Impuls umgewandelt. 
Um nun die Lebensdauer der Myonen zumessen werden die Myonen benötigt, die ihre gesamte kinetische Energie verlieren und somit im Szintillator verbleiben und dort Zerfallen.
Das Myon löst somit ein Signal aus wenn es in den Szintillator eintritt aber auch das Elektron das beim Zerfall entsteht löst eins aus.
Die Zeit zwischen diesen beiden Signalen ist also die Lebensdauer eines Myons.

\subsection{Lebensdauer}
Die Lebensdauer von Myonen ist ein stochastischer Prozess. 
Jedes einzelne Myon hat eine unterschiedliche Lebensdauer. Daher wird die Wahrscheinlichkeit dW, das ein Myon im infinitesimalen Bereich zerfällt benötigt. 
Es lässt sich der Zusammenhang

\begin{align*}
dW = \lambda dt
\end{align*}

\noindent herstellen. Dabei ist $\lambda$ die charakteristische Zerfallskonstante. 
Also hängt die Zerfallswahrscheinlichkeit nicht vom Alter eins individuellen Teilchens ab. 
Das Myon unterliegt keinen Alterungsprozess. Daraus ergibt sich der weitern Zusammenhang
\begin{align*}
dN = -N dW =-\lambda N dt
\end{align*}

\noindent mit $dN$ der Zahl der Teilchen, die im Zeitraum dt Zerfallen sind, wenn die Anzahl $N$ Teilchen beobachtet werden.  
Für die Lebensdauer folgt daraus die Exponentialverteilung auf dem Intervall $t$ bis $dt$.

\begin{align*}
dN(t) = N_0 \lambda e^- \lambda t dt
\end{align*}

\noindent Der Erwartungswert von dieser Verteilung ist die charakteristische Lebensdauer der Myonen $\tau$.

\begin{align*}
\tau = \int_{0}^{1} \lambda t e^-\lambda \, \symup{d}t = \frac{1}{\lambda}
\end{align*}
