\section{Theorie}
\label{sec:Theorie}

\subsection{Zielsetzung}
In diesen Versuch geht es um das allgemeien Relaxationsverhalten eines RC-Kreises.

\subsection{Relaxationsverhalten des RC-Kreises}
Eine Relaxionserscheinung beschreibt das Verhalten eines Systemes, dass seinen Ausgangszustand zunächst verlässt und 
dann nicht-oszillatorisch in seinen Ausgangszustand zurückkehrt.
Dabei ist die Geschwindigkeit der Änderung einer physikalischen Größe $A$ proportional zur 
Abweichung der Größe $A$.


\subsection{Auf- und Entladevorgang im RC-Kreis}
Ein Beispiel für so einen Vorgang sind die Aufladung und Entladung eines Kondensators über einen 
Widerstand $R$.

Dabei liegt zwischen den Platten eines Kondesators die Spannung 
    \begin{equation}
        \label{eq:sp}
        U_C = \dfrac{Q}{C}
    \end{equation} 
\noindent an.
Daraus lässt sich für den Strom die Gleichung 
   \begin{equation}
       \label{eq:st}
       I = \dfrac{U_C}{R}.
   \end{equation}
\noindent herleiten.

\noindent Es fließt über das Zeitintervall $dt$ die Ladung $I dt$ über. Also ändert sich die Ladung der Platten
des Kondensators mit $dQ=-I\ dt$ .

\noindent Aus den aufgeführten Gleichung kann nun eine Differentialgleichung für den zeitlichen 
Verlauf der Ladung des Kondensators bestimmt werden.

\begin{equation}
    \dfrac{dQ}{dt}= -\dfrac{1}{RC} Q{t}
\end{equation}

\noindent Die Ladung auf dem Kondensator konvergiert für große Zeit gegen 0. Daher wird durch 
Integration der Entladevorgang durch 
\begin{equation}
    \label{eq:dgl}
    Q(t)=Q(0)e^{\dfrac{-t}{RC}}
\end{equation}
\noindent beschrieben.


\noindent Für den Aufladevorgang gelten die Randbedingung  $Q(0)=0$ und $Q(\infty)=CU_0$. 
Daher lässt sich die Differentialgleichung durch 

\begin{equation}
    Q(t) = CU_0(1-e^{\dfrac{-t}{RC}})
\end{equation}
\noindent lösen.
\noindent Hier wird der Ausdruck $RC$ als Zeitkonstante bezeichnet. Sie ist ein Maß für die Relaxationsgeschwindigkeit.
Während des Zeitintervalls $\Delta T=RC$ sinkt die Ladung des Kondensators um den Faktor
 $\dfrac{Q(t=RC)}{Q(0)}=\dfrac{1}{e}$.

\noindent Nach $\Delta T = 2,3RC$ sind 10 \% des Ausgangswertes und nach $\Delta T = 4,6RC$ sind 1\% erreicht.

\subsection{Relaxationsverhalten unter Einfluss angelegter periodischer Spannung}
Ein mechanisches System, das unter Einfluss einer Kraft mit sinusförmiger Zeitabhängigkeit ist, lässt sich analog zu einem RC-Kreis mit 
einer Sinusspannung beschreiben.
Die Spannung $U(t)$ kann beschrieben werden durch 
\begin{equation*}
    U(t) = U_0 \text{cos}( \omega t) .
\end{equation*}

\noindent Wenn die Kreisfrequenz $\omega \ll 1/RC$  ist wird die Spannung am Kondensator gleich der Wechselspannung.
Bei höherer Kreisfrequenz bildet sich zwischen den beiden Spannungen eine Phasenverschiebung $\phi$ und die Amplitude 
der Kondensatorspannung $U_c$ wird kleiner.

Nun wird der Frequenzabhängigkeit näher betrachtet.
Als Ansatz wird

\begin{equation}
    U_{\text{C}}(t) = A(\omega) \text{cos}(\omega t + \phi \{ \omega \} )
    \label{eqn:7}
\end{equation}

\noindent gewählt.

%\begin{figure}[H]
%    \centering
%    \includegraphics{schalt3.png}
%    \caption{Der RC-Schaltkreis mit angelegter Cosinus-Spannung}
%    \label{fig:schalt3}
%\end{figure}

\noindent Die Spannung, die in der Abbildung  \ref{fig:schalt3} dargestellten Stromkreises lässt sich  mittels des zweiten Kirchhoffschen Gesetz
aufstellen.:
\begin{equation}
    \label{eq:kirsche}
    U(t)=U_R(t)+U_C(t) \Leftrightarrow U_0 \cos{\omega t}=I(t)R + 
    A(\omega)\cos{\omega t + \phi}
\end{equation}
\noindent Außerdem kann der Strom $I(t)$ durch 
\begin{equation}
    I(t) = \frac{\text{d}Q}{\text{d}t} = C \frac{\text{d}U_\text{C}}{\text{d}t}
    \label{eqn:9}
\end{equation}
\noindent dargestellt werden. Mit  $\omega t = \frac{\pi}{2}$ und der Gleichung \ref{eq:kirsche} lässt sich 
\begin{equation*}
    0 = -\omega R C \text{sin} \left( \frac{\pi}{2} + \phi \right) + \text{cos} \left( \frac{\pi}{2} + \phi \right)
\end{equation*}

\noindent aufstellen.
Daraus folgt wegen $\text{sin}(\phi + \pi /2 = \text{cos}(\phi))$ und $\text{cos}(\phi + \pi/2 = - \text{sin}
(\phi))$

\begin{equation}
    \phi (\omega) = \text{arctan} ( - \omega R C) .
    \label{eqn:11}
\end{equation}

\noindent Es zeigt auf, dass $\phi$ für niedrige Werte gegen Null geht  und sich für hohe Frequenzen den Wert $\frac{\pi}{2}$ nähert.
Für $\omega = \frac{1}{RC}$ ist $\phi = \frac{\pi}{4}$.

Es folgt für $\omega t + \phi = \frac{\pi}{2}$ aus der Gleichung \ref{eq:kirsche}

\begin{equation}
    A(\omega) = - \frac{\text{sin} \phi}{\omega R C} U_0 .
    \label{eqn:12}
\end{equation}

\noindent Aus dieer Gleichung lässt sich nun mithilfe der Beziehung $\text{sin}^2 \phi + \text{cos}^2 \phi = 1$  
\begin{equation*}
    A(\omega) = \frac{U_0}{\sqrt{1 + \omega^2 R^2 C^2}} .
\end{equation*}
\noindent herleiten.
Diese beschreibt die Beziehung zwischen der Amplitude der Kondensatorspannung und der Kreisfrequenz der erreger Spannung.
Es wird aufgezeigt, dass für $\omega \rightarrow 0$ $A(\omega)$ gegen $U(t)$ geht und für $\omega \rightarrow \infty$ verschwindet
die Amplitude.
Daher ist der RC-Kreis ein guter Tiefpass.

\subsection{Der RC-Kreis als Integrator}
Ein weiterer Effekt des RC-Kreises ist die Integration der Spannung. Dafür  muss die Frequenz $\omega \gg \frac{1}{RC}$ und 
$|U_{\text{c}}| \ll | U_{\text{R}}|$.
Mithilfe der Gleichung \ref{eqn:9} und 

\begin{equation*}
    U(t) = I(t) R + U_{\text{C}}(t)
\end{equation*}

\noindent lässt sich 
\begin{equation*}
    U_{\text{C}}(t) = \frac{1}{RC} \int_0^t U(t')\text{d}t' .
\end{equation*}

\noindent herleiten.